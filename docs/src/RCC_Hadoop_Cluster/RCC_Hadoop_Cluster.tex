\documentclass{beamer}

\usepackage[utf8]{inputenc}
\usepackage{default}

\mode<presentation>
%{ \usetheme{boxes} }


\usetheme{Madrid}

\usepackage{times}
\usepackage{graphicx}
\usepackage{tabulary}
\usepackage{listings}
\usepackage{verbatimbox}
\usepackage{graphicx}
\usepackage{lmodern}
\usepackage[absolute,overlay]{textpos}
\usepackage{pgfpages}
\usepackage{color}

\pgfdeclareimage[height=1.0cm]{logo_rcc}{../icons/logo_rcc.png}
\setlength{\TPHorizModule}{1mm}
\setlength{\TPVertModule}{1mm}
\newcommand{\RCCLogo}{
\begin{textblock}{14}(1.5,1.5)
  \pgfuseimage{logo_rcc}
\end{textblock}
}

\definecolor{mycolorcli}{RGB}{53,154,26}
\definecolor{mycolorcode}{RGB}{0,0,255}
\definecolor{mycolordef}{RGB}{255,0,0}
\definecolor{mycolorlink}{RGB}{184,4,255}

\title{\huge{RCC Hadoop Cluster}}
\author{Igor Yakushin \\ \texttt{ivy2@uchicago.edu}}
\date{}

\definecolor{ChicagoMaroon}{RGB}{128,0,0}

\setbeamercolor{title}{bg=ChicagoMaroon}

\begin{document}

\setbeamertemplate{navigation symbols}{}

\setbeamercolor{fcolor}{fg=white,bg=ChicagoMaroon}
\setbeamertemplate{footline}{
\begin{beamercolorbox}[ht=4ex,leftskip=1.4cm,rightskip=.3cm]{fcolor}
\hrule
\vspace{0.1cm}
   \hfill \insertshortdate \hfill \insertframenumber/\inserttotalframenumber
\end{beamercolorbox}
}

\setbeamercolor{frametitle}{bg=ChicagoMaroon,fg=white}

\begin{frame}
\titlepage
\end{frame}




\section{RCC Hadoop cluster}
\begin{frame}
  \frametitle{RCC Hadoop cluster}
  \begin{itemize}
  \item RCC Hadoop cluster was paid for by MSCA program in 2017 and is managed by Research Computing Center
  \item Currently it consists of 3 management nodes, 6 data nodes, the rack and 10G switch.
  \item On the internal network the corresponding host names are {\color{mycolorcli}md01-3}, {\color{mycolorcli}hd01-6}
  \item Originally there were 4 data nodes, 2 more were added in 2019
  \item The total cost of the cluster was about \$100k
  \item About \$5k per year is paid for Cloudera software support
  \item One of the management nodes also serves as a login node for users
  \item The login node's hostname on the internal network is {\color{mycolorcli}md01}, that's what you see when you log into it,
    and on the UChicago network it is {\color{mycolorcli}hadoop.rcc.uchicago.edu} - you use this address for ssh, jupyter hub, Hue
  \end{itemize} 
\end{frame}

\begin{frame}
  \frametitle{RCC Hadoop cluster}
  \begin{itemize}
  \item Each management node has 256G of RAM, two mirrored 300G SSDs for OS, 24 CPU cores
  \item Each data node has 256G of RAM, two mirrored 300G SSDs for OS, 24 CPU cores, 14 4T HDDs that are part of HDFS
  \item The total HDFS storage that is built on top of $6*14$ 4T disks spread accross data nodes is 300T
  \item Hadoop software consists of many different services running on management and data nodes.
  \item As a user, you only interact directly with a login node and submit a job to the cluster using YARN scheduler
  \item The actual computation is spread by YARN over data nodes in a way transparent to the user.
  \item The cluster is currently running Cloudera 6.3 distribution
  \end{itemize}
\end{frame}


\begin{frame}
  \frametitle{RCC Hadoop cluster}
  \begin{itemize}
  \item The nodes are connected to each other with 10G ethernet network
  \item Each node is running CentOS 7.6 distribution of Linux. This is slightly different from Scientific Linux 7.4 used on midway
    but for a user there is very little difference: all the same commands should work on both systems
  \item The only thing to be careful about: if you build a binary on one cluster it might not necessarily run on the other cluster
    because the libraries that are loaded at run time might not be available or might be of incompatible version.
  \item Same applies to python: despite the fact that python programs are text files and are not explicitly compiled by a user,
    under the hood, when you install a python package on your account, sometimes some compilation might happen.
  \end{itemize}
\end{frame}


\begin{frame}
  \frametitle{RCC Hadoop cluster}
  \begin{itemize}
  \item Besides HDFS and the local file system on which OS is installed, the same GPFS file systems as on midway are mounted on each node:
    \begin{itemize}
    \item {\color{mycolorcli}/home}
    \item {\color{mycolorcli}/project2}
    \item {\color{mycolorcli}/scratch/midway2}
    \item {\color{mycolorcli}/software}
    \end{itemize}
  \item In particular, when you login, you get into the same home file system as on midway. Midway, however, does not have access to HDFS.
  \item So, if you need to copy data between GPFS and HDFS, do it on Hadoop cluster using HDFS commands
  \item Hadoop cluster is only visible on UChicago network. When you are off-campus, you need to use VPN to access it.
  \end{itemize}
\end{frame}

\end{document}


