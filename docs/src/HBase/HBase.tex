\documentclass{beamer}

\usepackage[utf8]{inputenc}
\usepackage{default}

\mode<presentation>
%{ \usetheme{boxes} }


\usetheme{Madrid}

\usepackage{times}
\usepackage{graphicx}
\usepackage{tabulary}
\usepackage{listings}
\usepackage{verbatimbox}
\usepackage{graphicx}
\usepackage{lmodern}
\usepackage[absolute,overlay]{textpos}
\usepackage{pgfpages}
\usepackage{color}

\pgfdeclareimage[height=1.0cm]{logo_rcc}{../icons/logo_rcc.png}
\setlength{\TPHorizModule}{1mm}
\setlength{\TPVertModule}{1mm}
\newcommand{\RCCLogo}{
\begin{textblock}{14}(1.5,1.5)
  \pgfuseimage{logo_rcc}
\end{textblock}
}

\definecolor{mycolorcli}{RGB}{53,154,26}
\definecolor{mycolorcode}{RGB}{0,0,255}
\definecolor{mycolordef}{RGB}{255,0,0}
\definecolor{mycolorlink}{RGB}{184,4,255}

\title{\huge{HBase}}
\author{Igor Yakushin \\ \texttt{ivy2@uchicago.edu}}
%\date{January 20, 2018}

\definecolor{ChicagoMaroon}{RGB}{128,0,0}

\setbeamercolor{title}{bg=ChicagoMaroon}

\begin{document}

\setbeamertemplate{navigation symbols}{}

\setbeamercolor{fcolor}{fg=white,bg=ChicagoMaroon}
\setbeamertemplate{footline}{
\begin{beamercolorbox}[ht=4ex,leftskip=1.4cm,rightskip=.3cm]{fcolor}
\hrule
\vspace{0.1cm}
   \hfill \insertshortdate \hfill \insertframenumber/\inserttotalframenumber
\end{beamercolorbox}
}

\setbeamercolor{frametitle}{bg=ChicagoMaroon,fg=white}

\begin{frame}
\titlepage
\end{frame}


\section{Introduction}
\begin{frame}
  \frametitle{Introduction}
  \begin{itemize}
  \item {\color{mycolordef}\textbf{H}}adoop data{\color{mycolordef}\textbf{Base}}
  \item Consists of tables
  \item Each table is sparse, distributed, persistent, multidimensional sorted map, indexed by rowkey, column family, column, timestamp
  \item Can store structured, semistructured, unstructured data
  \item Does not care about types
  \item Not a relational database, does not speak SQL natively, does not enforce relationship in data
  \item Designed to run on a cluster of computers, scale horizontally as you add more machines to the cluster
  \item The main operations are: create (table), put (value into cell), get (value from cell), scan (values from cells)
  \item Various auxiliary operations: alter, list, describe, ...
  \end{itemize}
\end{frame}

\section{Column families, columns, multiple versions}
\begin{frame}[fragile]
 \frametitle{Column families, columns, multiple versions}
 {\color{mycolorcode}
 \begin{verbnobox}[\tiny]
  Row Key     Column Family: {Column Qualifier:Version:Value}
  ------------------------------------------------------------------------
  00001       CustomerName:  {‘FN’: 1383859182496:‘John’,
                             ‘LN’: 1383859182858:‘Smith’,
                             ‘MN’: 1383859183001:’Timothy’,
                             ‘MN’: 1383859182915:’T’}
              ContactInfo:   {‘EA’: 1383859183030:‘John.Smith@xyz.com’,
                            ’SA’: 1383859183073:’1 Hadoop Lane, NY 11111’}
  00002       CustomerName:  {‘FN’: 1383859183103:‘Jane’,
                             ‘LN’: 1383859183163:‘Doe’,
              ContactInfo:   {’SA’: 1383859185577:’7 HBase Ave, CA 22222’}
 \end{verbnobox}
 }
 \begin{itemize}
 \item Internally HBase table is stored in {\color{mycolordef}HFiles} - different set of files for different column families
 \item HFiles for the same column family are periodically merged together or split and
   distributed among the nodes to maintain high performance and fault-taulerance
 \item On top of HDFS
 \item One can specify how many latest versions of data to keep in HBase table or to query versions in a particular date-time range
\end{itemize}
\end{frame}

\section{HBase shell}
\begin{frame}[fragile]
 \frametitle{HBase shell}
{\color{mycolorcli}
  \begin{lstlisting}[frame=single, basicstyle=\tiny]
$ hbase shell

> create 'CustomerContactInfo', 'CustomerName', 'ContactInfo'
> put 'CustomerContactInfo', '00001', 'CustomerName:FN', 'John'
> put 'CustomerContactInfo', '00001', 'CustomerName:LN', 'Smith'
> put 'CustomerContactInfo', '00001', 'CustomerName:MN', 'T'
> put 'CustomerContactInfo', '00001', 'ContactInfo:EA', 'John.Smith@xyz.com'
> put 'CustomerContactInfo', '00001', 'ContactInfo:SA', '1 Hadoop Lane, NY 11111'
> put 'CustomerContactInfo', '00002', 'CustomerName:FN', 'Jane'
> put 'CustomerContactInfo', '00002', 'CustomerName:LN', 'Doe'
> put 'CustomerContactInfo', '00002', 'ContactInfo:SA', '7 HBase Ave, CA 22222'
>list
=> ["CustomerContactInfo"]
>describe 'CustomerContactInfo'
Table CustomerContactInfo is ENABLED
CustomerContactInfo
COLUMN FAMILIES DESCRIPTION                                     
{NAME => 'ContactInfo', DATA_BLOCK_ENCODING => 'NONE', BLOOMFILTER => 'ROW', 
REPLICATION_SCOPE => '0', VERSIONS => '1', COMPRESSION => 'NONE', 
MIN_VERSIONS => '0', TTL => 'FOREVER', KEEP_DELETED_CELLS => 'FALSE', 
BLOCKSIZE => '65536', IN_MEMORY => 'false', BLOCKCACHE => 'true'}           
{NAME => 'CustomerName', DATA_BLOCK_ENCODING => 'NONE', BLOOMFILTER => 'ROW', 
REPLICATION_SCOPE => '0', VERSIONS => '1', COMPRESSION => 'NONE', MIN_VERSIONS => '0', 
TTL => 'FOREVER', KEEP_DELETED_CELLS => 'FALSE', BLOCKSIZE => '65536', 
IN_MEMORY => 'false', BLOCKCACHE => 'true'}                                     
  \end{lstlisting}
}
\end{frame}

\begin{frame}[fragile]
 \frametitle{HBase shell}
{\color{mycolorcli}
  \begin{lstlisting}[frame=single, basicstyle=\tiny]

> alter 'CustomerContactInfo', NAME => 'CustomerName', VERSIONS => 5
> describe 'CustomerContactInfo'
Table CustomerContactInfo is ENABLED                                     
CustomerContactInfo
COLUMN FAMILIES DESCRIPTION
{NAME => 'ContactInfo', DATA_BLOCK_ENCODING => 'NONE', BLOOMFILTER => 'ROW', 
REPLICATION_SCOPE => '0', VERSIONS => '1', COMPRESSION => 'NONE', 
MIN_VERSIONS => '0', TTL => 'FOREVER', KEEP_DELETED_CELLS => 'FALSE', 
BLOCKSIZE => '65536', IN_MEMORY => 'false', BLOCKCACHE => 'true'}                                     
{NAME => 'CustomerName', DATA_BLOCK_ENCODING => 'NONE', BLOOMFILTER => 'ROW', 
REPLICATION_SCOPE => '0', VERSIONS => '5', COMPRESSION => 'NONE', 
MIN_VERSIONS => '0', TTL => 'FOREVER', KEEP_DELETED_CELLS => 'FALSE', 
BLOCKSIZE => '65536', IN_MEMORY => 'false', BLOCKCACHE => 'true'}
> put 'CustomerContactInfo', '00001', 'CustomerName:MN', 'Timothy'
> scan 'CustomerContactInfo', {VERSIONS => 2}
ROW      COLUMN+CELL 
00001    column=ContactInfo:EA, timestamp=1471196578957, value=John.Smith@xyz.com 
00001    column=ContactInfo:SA, timestamp=1471196578988, value=1 Hadoop Lane, NY 11111 
00001    column=CustomerName:FN, timestamp=1471196578805, value=John 
00001    column=CustomerName:LN, timestamp=1471196578859, value=Smith 
00001    column=CustomerName:MN, timestamp=1471197270641, value=Timothy 
00001    column=CustomerName:MN, timestamp=1471196578901, value=T 
00002    column=ContactInfo:SA, timestamp=1471196579070, value=7 HBase Ave, CA 22222 
00002    column=CustomerName:FN, timestamp=1471196579016, value=Jane 
00002    column=CustomerName:LN, timestamp=1471196579042, value=Doe
  \end{lstlisting}
}
\end{frame}

\begin{frame}[fragile]
 \frametitle{HBase shell}
{\color{mycolorcli}
  \begin{lstlisting}[frame=single, basicstyle=\tiny]
> get 'CustomerContactInfo', '00001'
COLUMN                CELL 
ContactInfo:EA        timestamp=1471196578957, value=John.Smith@xyz.com 
ContactInfo:SA        timestamp=1471196578988, value=1 Hadoop Lane, NY 11111 
CustomerName:FN       timestamp=1471196578805, value=John 
CustomerName:LN       timestamp=1471196578859, value=Smith 
CustomerName:MN       timestamp=1471197270641, value=Timothy
> get 'CustomerContactInfo', '00001', {COLUMN => 'CustomerName:MN'}
COLUMN                CELL 
CustomerName:MN       timestamp=1471197270641, value=Timothy
> disable 'CustomerContactInfo'
> drop 'CustomerContactInfo'
> quit
  \end{lstlisting}
}

\end{frame}


\section{Clients}
\begin{frame}
 \frametitle{HBase: clients}
 Besides HBase shell, one can use HBase with
 \begin{itemize}
  \item MapReduce JavaAPI
  \item Hive
  \item Pig
  \item Spark
  \item Impala
  \end{itemize}
\end{frame}


\section{HBase applications}
\begin{frame}
  \frametitle{HBase applications}
  \begin{itemize}
    \item For what applications is HBase good for?
      \begin{itemize}
      \item Huge amount of (semi-)structured data: most records have the same columns but some might have extra columns
      \item The need for random and real time access to data
      \item The need to store multiple versions of records
      \item No support for relational features
      \item No support for transactions
      \end{itemize}
    \item Example applications:
      \begin{itemize}
      \item  Mozilla stores crash data in HBase
      \item Facebook uses HBase to store real-time messages
      \end{itemize}
  \end{itemize}
\end{frame}

\end{document}
