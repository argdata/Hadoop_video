\documentclass{beamer}

\usepackage[utf8]{inputenc}
\usepackage{default}

\mode<presentation>
%{ \usetheme{boxes} }


\usetheme{Madrid}

\usepackage{times}
\usepackage{graphicx}
\usepackage{tabulary}
\usepackage{listings}
\usepackage{verbatimbox}
\usepackage{graphicx}
\usepackage{lmodern}
\usepackage[absolute,overlay]{textpos}
\usepackage{pgfpages}
\usepackage{color}
\usepackage{multicol}

\pgfdeclareimage[height=1.0cm]{spark}{../../icons/spark.png}
\newcommand{\SPARK}{
\begin{textblock}{14}(107.5,1.5)
  \pgfuseimage{spark}
\end{textblock}
}



\definecolor{mycolorcli}{RGB}{53,154,26}
\definecolor{mycolorcode}{RGB}{0,0,255}
\definecolor{mycolordef}{RGB}{255,0,0}
\definecolor{mycolorlink}{RGB}{184,4,255}

\setcounter{tocdepth}{3}

\title{\huge{BigDL}}
\author{Igor Yakushin \\ \texttt{ivy2@uchicago.edu}}

\definecolor{ChicagoMaroon}{RGB}{128,0,0}

\setbeamercolor{title}{bg=ChicagoMaroon}

\begin{document}

\setbeamertemplate{navigation symbols}{}

\setbeamercolor{fcolor}{fg=white,bg=ChicagoMaroon}
\setbeamertemplate{footline}{
\begin{beamercolorbox}[ht=4ex,leftskip=1.4cm,rightskip=.3cm]{fcolor}
\hrule
\vspace{0.1cm}
   \hfill \insertshortdate \hfill \insertframenumber/\inserttotalframenumber
\end{beamercolorbox}
}

\setbeamercolor{frametitle}{bg=ChicagoMaroon,fg=white}

\begin{frame}
\SPARK
\titlepage
\end{frame}

\section{Introduction}
\begin{frame}[fragile]
  \frametitle{Introduction}
  \begin{itemize}
  \item {\color{mycolordef}BigDL} is a library for Deep Learning from Intel
  \item It is fully integrated with Spark and can work directly with RDDs
  \item While MLlib has some machine learning algorithms, 
    including fully connected Neural Networks, it does not currently implement Deep Learning models
  \item BigDL runs on CPU only and relies on MKL for multithreading
  \item The installation of python version of BigDL is as simple as 
    {\color{mycolorcli}
\begin{verbatim}
pip install BigDL
\end{verbatim}
    }
  \item BigDL API appears to be very similar to Keras
  \end{itemize}
\end{frame}


\section{LeNet model}
\begin{frame}[fragile]
  \frametitle{LeNet model}
  \begin{itemize}
  \item In this lab we use the models supplied with BigDL
  \item {\color{mycolorcli}\verb|local_lenet.py|} is a standalone (no Spark) 
    BigDL program to train LeNet model for handwritten digits recognition
  \item Run it on a single compute node with
    {\color{mycolorcli}
\begin{verbatim}
make lenet_local
\end{verbatim}
    }
  \item {\color{mycolorcli}lenet5.py} is using both Spark and BigDL.
  \item Run it on a single compute node with
    {\color{mycolorcli}
\begin{verbatim}
make lenet5
\end{verbatim}
    }
\end{itemize}
\end{frame}


\begin{frame}[fragile]
  \frametitle{LeNet model}
  \begin{itemize}
  \item LeNet has been used for decades by USPS to recognize handwritten zip codes
  \item Here is how LeNet model is built in BigDL:
{\small
    {\color{mycolorcode}
\begin{verbatim}
def build_model(class_num):
    model = Sequential()
    model.add(Reshape([1, 28, 28]))
    model.add(SpatialConvolution(1, 6, 5, 5))
    model.add(Tanh())
    model.add(SpatialMaxPooling(2, 2, 2, 2))
    model.add(Tanh())
    model.add(SpatialConvolution(6, 12, 5, 5))
    model.add(SpatialMaxPooling(2, 2, 2, 2))
    model.add(Reshape([12 * 4 * 4]))
    model.add(Linear(12 * 4 * 4, 100))
    model.add(Tanh())
    model.add(Linear(100, class_num))
    model.add(LogSoftMax())
    return model
\end{verbatim}
    }
}
\end{itemize}
\end{frame}


\begin{frame}[fragile]
  \frametitle{LeNet model}
  \begin{itemize}
  \item MNIST data is loaded into RDD and is preprocessed in Spark before going into BigDL:
{\small
    {\color{mycolorcode}
\begin{verbatim}
train_data = get_mnist(sc, "train", options.dataPath)\
     .map(lambda rec_tuple: (normalizer(rec_tuple[0],\
      mnist.TRAIN_MEAN, mnist.TRAIN_STD), rec_tuple[1]))\
     .map(lambda t: Sample.from_ndarray(t[0], t[1]))
\end{verbatim}
}
}
\end{itemize}
\end{frame}

\end{document}
