\documentclass{beamer}

\usepackage[utf8]{inputenc}
\usepackage{default}

\mode<presentation>
%{ \usetheme{boxes} }


\usetheme{Madrid}

\usepackage{times}
\usepackage{graphicx}
\usepackage{tabulary}
\usepackage{listings}
\usepackage{verbatimbox}
\usepackage{graphicx}
\usepackage{lmodern}
\usepackage[absolute,overlay]{textpos}
\usepackage{pgfpages}
\usepackage{color}

\pgfdeclareimage[height=1.0cm]{logo_rcc}{../icons/logo_rcc.png}
\setlength{\TPHorizModule}{1mm}
\setlength{\TPVertModule}{1mm}
\newcommand{\RCCLogo}{
\begin{textblock}{14}(1.5,1.5)
  \pgfuseimage{logo_rcc}
\end{textblock}
}

\definecolor{mycolorcli}{RGB}{53,154,26}
\definecolor{mycolorcode}{RGB}{0,0,255}
\definecolor{mycolordef}{RGB}{255,0,0}
\definecolor{mycolorlink}{RGB}{184,4,255}

\title{\huge{HDFS}}
\author{Igor Yakushin \\ \texttt{ivy2@uchicago.edu}}
%\date{January 20, 2018}

\definecolor{ChicagoMaroon}{RGB}{128,0,0}

\setbeamercolor{title}{bg=ChicagoMaroon}

\begin{document}

\setbeamertemplate{navigation symbols}{}

\setbeamercolor{fcolor}{fg=white,bg=ChicagoMaroon}
\setbeamertemplate{footline}{
\begin{beamercolorbox}[ht=4ex,leftskip=1.4cm,rightskip=.3cm]{fcolor}
\hrule
\vspace{0.1cm}
   \hfill \insertshortdate \hfill \insertframenumber/\inserttotalframenumber
\end{beamercolorbox}
}

\setbeamercolor{frametitle}{bg=ChicagoMaroon,fg=white}

\begin{frame}
\titlepage
\end{frame}




\section{HDFS}
\subsection{Introduction}
\begin{frame}
 \frametitle{HDFS}
    \begin{itemize}
     \item Any file that is put into HDFS is automatically split into blocks (by default each block is 128M)
     \item Each block is replicated (by default 3 times)
     \item The blocks are spread across the cluster so that
      \begin{itemize}
	\item calculations can be done in parallel on different blocks to increase performance
	\item the file system is fault-taulerant with respect to disk/node/rack failure
	\item if a node goes down, HDFS takes care of creating more replicas automatically
	\item when a node comes back, extra replicas are destroyed
	\item if a new node is added, data spreads on it automatically
	\item there is a {\color{mycolordef}NameNode} that keeps track of where the blocks are
      \end{itemize}
	\item The calculations are done on {\color{mycolordef}DataNodes}
        \item Hadoop tries to do computations where the data is to minimize communication between nodes which is much slower
    \end{itemize} 
\end{frame}

\subsection{Usage}
\begin{frame}[fragile]
  \frametitle{HDFS: user interface}
{\color{mycolorcli}
  \begin{lstlisting}[frame=single, basicstyle=\tiny]
$ hdfs dfs -ls /user/$USER
drwx------   - ivy2 ivy2          0 2016-08-12 20:11 /user/ivy2/.Trash                                                                                                                                                                     

$ hdfs dfs -mkdir /user/$USER/test2 

$ hdfs dfs -put /usr/share/dict/linux.words /user/$USER/test2/

$ hdfs dfs -ls -h  /user/$USER/test2/
-rw-r--r--   3 ivy2 ivy2      4.7 M 2016-08-12 20:12 /user/ivy2/test2/linux.words

$ hdfs dfs -setrep 4  /user/$USER/test2/linux.words
Replication 4 set: /user/ivy2/test2/linux.words

$ hdfs dfs -ls -h  /user/$USER/test2/
-rw-r--r--   4 ivy2 ivy2      4.7 M 2016-08-12 20:12 /user/ivy2/test2/linux.words

$ hdfs dfs -mv /user/$USER/test2/linux.words /user/$USER/test2/words.txt

$ hdfs dfs -get /user/$USER/test2/words.txt

$ hdfs dfs -rm /user/$USER/test2/words.txt
  \end{lstlisting}
}


\end{frame}


\end{document}
