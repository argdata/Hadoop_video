\documentclass{beamer}

\usepackage[utf8]{inputenc}
\usepackage{default}

\mode<presentation>
%{ \usetheme{boxes} }


\usetheme{Madrid}

\usepackage{times}
\usepackage{graphicx}
\usepackage{tabulary}
\usepackage{listings}
\usepackage{verbatimbox}
\usepackage{graphicx}
\usepackage{lmodern}
\usepackage[absolute,overlay]{textpos}
\usepackage{pgfpages}
\usepackage{color}
\usepackage{multicol}


\pgfdeclareimage[height=1.0cm]{spark}{../../icons/spark.png}
\newcommand{\SPARK}{
\begin{textblock}{14}(107.5,1.5)
  \pgfuseimage{spark}
\end{textblock}
}

\definecolor{mycolorcli}{RGB}{53,154,26}
\definecolor{mycolorcode}{RGB}{0,0,255}
\definecolor{mycolordef}{RGB}{255,0,0}
\definecolor{mycolorlink}{RGB}{184,4,255}

\setcounter{tocdepth}{3}

\title{\huge{Spark SQL}}
\author{Igor Yakushin \\ \texttt{ivy2@uchicago.edu}}

\definecolor{ChicagoMaroon}{RGB}{128,0,0}

\setbeamercolor{title}{bg=ChicagoMaroon}

\begin{document}

\setbeamertemplate{navigation symbols}{}

\setbeamercolor{fcolor}{fg=white,bg=ChicagoMaroon}
\setbeamertemplate{footline}{
\begin{beamercolorbox}[ht=4ex,leftskip=1.4cm,rightskip=.3cm]{fcolor}
\hrule
\vspace{0.1cm}
   \hfill \insertshortdate \hfill \insertframenumber/\inserttotalframenumber
\end{beamercolorbox}
}

\setbeamercolor{frametitle}{bg=ChicagoMaroon,fg=white}

\begin{frame}
\SPARK
\titlepage
\end{frame}

\begin{frame}[fragile]
  \frametitle{Spark SQL}
  
  \begin{itemize}
  \item {\color{mycolordef}Spark SQL} is a component on top of Spark that introduced a data abstraction called {\color{mycolordef}DataFrames}, which provides support for structured and semi-structured data. 
  \item DataFrame is like a table distributed over the cluster similar to RDD
  \item One can query DataFrame using 
    \begin{itemize}
    \item Spark language
    \item SQL - language for queries on relational databases
    \end{itemize}
  \item In both cases the query optimization is used that typically results in a better performance over RDDs
  \item One can create DataFrame by applying transformations to other DataFrames, from the same sources as RDD: RDD, text files, json files, arrays, files in various Hadoop formats like parquet.
  \item Spark SQL can be interfaced with relational databases via ODBC/JDBC
  \item Spark SQL can use distributed Thrift store via ODBC/JDBC
  \item Spark SQL can use Hive store
  \end{itemize}
\end{frame}

\subsection{Lab 1}
\subsubsection{line\_count\_DF.py}
\begin{frame}[fragile]
  \frametitle{Spark SQL: Lab 1}
{\small
{\color{mycolorcode}
\begin{verbatim}
from pyspark import SparkContext, SparkConf
from pyspark.sql import SparkSession

sc = SparkContext(conf=SparkConf())
spark = SparkSession(sc)

inputData = spark.read.text(inputFile).cache()

numAs = inputData.filter(inputData.value.contains('a')).count()
numBs = inputData.filter(inputData.value.contains('b')).count()

print("Lines with a: %i, lines with b: %i" % (numAs, numBs))
\end{verbatim}
}
}
\end{frame}

\subsection{Lab 3}
\subsubsection{lab3.ipynb}
\begin{frame}[fragile]
  \frametitle{Spark SQL: Lab 3: lab3.ipynb}
  We use jupyter notebook:
{\color{mycolorcode}
\begin{verbatim}
lines = spark.read.text("README.md") 
from pyspark.sql.functions import *
z=lines.select(explode(split(lines.value,"\s+")).name("w"))
z.groupBy("w").count().take(10)
\end{verbatim}
}
\end{frame}


\subsection{Lab 4}
\subsubsection{sql.py}
\begin{frame}[fragile]
  \frametitle{Spark SQL: Lab 4: sql.py}
{\color{mycolorcode}
\begin{verbatim}
js = "data/people.json"
df = spark.read.json(js)
df.show()
df.printSchema()
df.select("name").show()
df.select(df['name'], df['age'] + 1).show()
df.filter(df['age'] > 21).show()
df.groupBy("age").count().show()

df.createOrReplaceTempView("people")
sqlDF = spark.sql("SELECT * FROM people")
sqlDF.show()
\end{verbatim}
}
\end{frame}

\subsubsection{rdd2df.py}
\begin{frame}[fragile]
  \frametitle{Spark SQL: Lab 4: rdd2df.py}
{\small
{\color{mycolorcode}
\begin{verbatim}
from pyspark.sql import SparkSession
from pyspark.sql import Row

spark = SparkSession.builder.getOrCreate()
sc = spark.sparkContext

lines = sc.textFile("data/people.txt")
parts = lines.map(lambda l: l.split(","))
people = parts.map(lambda p: Row(name=p[0], age=int(p[1])))

schemaPeople = spark.createDataFrame(people)
schemaPeople.createOrReplaceTempView("people")
teens = spark.sql("SELECT name FROM people 
                   WHERE age >= 13 AND age <= 19")
teenNames = teens.rdd.map(lambda p: "Name: " + p.name).collect()
for name in teenNames:
    print(name)
\end{verbatim}
}
}
\end{frame}

\subsubsection{files.py}
\begin{frame}[fragile]
  \frametitle{Spark SQL: Lab 4: files.py}
{\small
{\color{mycolorcode}
\begin{verbatim}
from pyspark.sql import SparkSession
from os.path import abspath
warehouse_location = abspath('spark-warehouse')
spark = SparkSession \
    .builder \
    .appName("Python Spark SQL Hive integration example") \
    .config("spark.sql.warehouse.dir", warehouse_location) \
    .enableHiveSupport() \
    .getOrCreate()
df = spark.read.load("data/users.parquet")
df.select("name", "favorite_color").write.save("output/t.parquet")

df = spark.sql("SELECT * FROM parquet.`data/users.parquet`")
print(df.show())

df.write.saveAsTable("users1")
df.write.option("path", "output/hive").saveAsTable("users")
\end{verbatim}
}
}
\end{frame}



\subsubsection{hive.py}
\begin{frame}[fragile]
  \frametitle{Spark SQL: Lab 4: hive.py}
{\small
{\color{mycolorcode}
\begin{verbatim}
warehouse_location = abspath('spark-warehouse')
spark = SparkSession \
    .builder \
    .appName("Python Spark SQL Hive integration example") \
    .config("spark.sql.warehouse.dir", warehouse_location) \
    .enableHiveSupport() \
    .getOrCreate()
spark.sql("CREATE TABLE IF NOT EXISTS 
           src (key INT, value STRING) USING hive")
spark.sql("LOAD DATA LOCAL INPATH 
           'data/kv1.txt' INTO TABLE src")
spark.sql("SELECT * FROM src").show()

spark.sql("SELECT * FROM users1").show()
\end{verbatim}
}
}
\end{frame}


\subsubsection{hive.sql}
\begin{frame}[fragile]
\frametitle{Spark SQL: Lab 4: hive.sql}
\begin{itemize}
\item hive.sql:
{\small
{\color{mycolorcode}
\begin{verbatim}
select * from users1;
\end{verbatim}
}
}

\item spark-sql:
{\small
{\color{mycolorcli}
\begin{verbatim}
spark-sql --master local[1] --name "spark-sql example" 
          -f hive.sql 1>spark_sql.out 2>spark_sql.err
\end{verbatim}
}
}
\end{itemize}
\end{frame}

\end{document}
