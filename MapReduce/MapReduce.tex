\documentclass{beamer}

\usepackage[utf8]{inputenc}
\usepackage{default}

\mode<presentation>
%{ \usetheme{boxes} }


\usetheme{Madrid}

\usepackage{times}
\usepackage{graphicx}
\usepackage{tabulary}
\usepackage{listings}
\usepackage{verbatimbox}
\usepackage{graphicx}
\usepackage{lmodern}
\usepackage[absolute,overlay]{textpos}
\usepackage{pgfpages}
\usepackage{color}

\pgfdeclareimage[height=1.0cm]{logo_rcc}{../icons/logo_rcc.png}
\setlength{\TPHorizModule}{1mm}
\setlength{\TPVertModule}{1mm}
\newcommand{\RCCLogo}{
\begin{textblock}{14}(1.5,1.5)
  \pgfuseimage{logo_rcc}
\end{textblock}
}

\definecolor{mycolorcli}{RGB}{53,154,26}
\definecolor{mycolorcode}{RGB}{0,0,255}
\definecolor{mycolordef}{RGB}{255,0,0}
\definecolor{mycolorlink}{RGB}{184,4,255}

\title{\huge{MapReduce}}
\author{Igor Yakushin \\ \texttt{ivy2@uchicago.edu}}
\date{}

\definecolor{ChicagoMaroon}{RGB}{128,0,0}

\setbeamercolor{title}{bg=ChicagoMaroon}

\begin{document}

\setbeamertemplate{navigation symbols}{}

\setbeamercolor{fcolor}{fg=white,bg=ChicagoMaroon}
\setbeamertemplate{footline}{
\begin{beamercolorbox}[ht=4ex,leftskip=1.4cm,rightskip=.3cm]{fcolor}
\hrule
\vspace{0.1cm}
   \hfill \insertshortdate \hfill \insertframenumber/\inserttotalframenumber
\end{beamercolorbox}
}

\setbeamercolor{frametitle}{bg=ChicagoMaroon,fg=white}

\begin{frame}
\titlepage
\end{frame}



\section{MapReduce}
\subsection{Introduction}
\begin{frame}
 \frametitle{MapReduce}
 \begin{itemize}
  \item {\color{mycolordef}\textbf{map}} - convert each record into $(key, value)$ pairs
  \item {\color{mycolordef}\textbf{reduce}} - apply some reduce operation to the resulting set of $(key, value)$; for example, sum values for each key, or find max, or min, etc.
  \item Obviously, maps can be done in parallel, independently from each other, on different nodes, for each record
  \item Between local map and global reduce, perhaps, apply reduce locally on each node before doing it globally and sort the results
  \item The native way to write your own MapReduce is to extend the corresponding Java classes from MapReduce API
  \item These days people rarely have to implement their own MapReduce but rather use high level tools like Pig, Hive, Spark, HBase, etc. unless you 
    are a programmer implementing such a high level tool
  \end{itemize}
\end{frame}

\subsection{Examples}
\begin{frame}
  \frametitle{Examples of applying MapReduce approach}
  \begin{itemize}
  \item {\color{mycolordef}Count how many times each word occurs in a text}
    \begin{itemize}
      \item map - for each word {\color{mycolorcode}$w_i$} in a record (for example, a line in a text file) output {\color{mycolorcode}$(w_i,1)$}
      \item reduce - for each {\color{mycolorcode}$w_i$} sum the values
    \end{itemize}
  \item {\color{mycolordef}Count the average number of social contacts a person has grouped by age}
    \begin{itemize}
    \item map - for each person {\color{mycolorcode}$p_i$}  of an age {\color{mycolorcode}$a_i$} in a record, output {\color{mycolorcode}$(a_i, (1, contacts_i))$}
    \item reduce - for each age {\color{mycolorcode}$a_i$}, sum separately the first and second component of the value to obtain number 
      of people of a given age and total number of contacts, divide the latter by the former
    \end{itemize}
  \end{itemize}
\end{frame}

\begin{frame}[fragile]
  \frametitle{Examples of applying MapReduce approach}
  \begin{itemize}
  \item {\color{mycolordef}Finding common friends, like in Facebook. } For each person a list of friends is given:
    \begin{lstlisting}[frame=single, basicstyle=\tiny]
      A -> B C D
      B -> A C D E
      C -> A B D E
      D -> A B C E
      E -> B C D
    \end{lstlisting}
    
    \begin{itemize}
    \item map - key is a person and a friend, sorted; value is the list of friends. For example, for C the output from map is:
      \begin{lstlisting}[frame=single, basicstyle=\tiny]
        (A C) -> A B D E
        (B C) -> A B D E
        (C D) -> A B D E
        (C E) -> A B D E
      \end{lstlisting}
    \item reduce - group the results by key, for example:
      \begin{lstlisting}[frame=single, basicstyle=\tiny]
        (A B) -> (A C D E) (B C D)
      \end{lstlisting}
      and find the intersection of value lists:
      \begin{lstlisting}[frame=single, basicstyle=\tiny]
        (A B) -> (C D)
      \end{lstlisting}
    \end{itemize}
  \end{itemize}
\end{frame}


\end{document}
